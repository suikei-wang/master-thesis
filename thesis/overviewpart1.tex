\chapter{An Overview of Numerical Optimization}
\label{cha:overviewpart1}
In this chapter, we aim to provide readers with an overview of numerical optimization. We begin with the theory of optimization (Section~\ref{sec:theory}), from the existence of optimizers, to the optimality conditions for both unconstrained and constrained problems with duality. As the theoretical background of optimization, this field provides a solid solution for the algorithm. \\
We then formally define the optimization of constrained problems in Section~\ref{sec:consopt} and describe the solution for these problems based on the gradient. \\
Next, we discuss briefly the bi-level optimization, which is a parameterized lower-level problem binds variables that appear in the objective of an upper-level problem (Section~\ref{sec:bilevel}). Finally, we give a summary of the numerical optimization in constrained problems in Section~\ref{sec:2summary}.




\section{Theory of Optimization}
\label{sec:theory}
\subsection{Existence of Optimizers}


\subsection{Optimality Conditions for Unconstrained Problems}

\subsection{Optimality Conditions for Constrained Problems}

\subsection{Duality}


\section{Constrained Optimization}
\label{sec:consopt}
\subsection{Equality Constrained Optimization}
\subsection{Inequality Constrained Optimization}

\section{Bi-level Optimization}
\label{sec:bilevel}



\section{Summary}
\label{sec:2summary}
Summary what you discussed in this chapter, and mention the story in next
chapter. Readers should roughly understand what your thesis takes about by only reading
words at the beginning and the end (Summary) of each chapter.



