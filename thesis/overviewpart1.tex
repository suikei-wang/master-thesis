\chapter{An Overview of Numerical Optimization}
\label{cha:overviewpart1}
In this chapter, we aim to provide readers with an overview of numerical optimization. We begin with the theory of optimization (Section~\ref{sec:theory}), from the existence of optimizers, to the optimality conditions for both unconstrained and constrained problems with duality. As the theoretical background of optimization, this field provides a solid solution for the algorithm. 
\par We then formally define the optimization of unconstrained and constrained problems in Section~\ref{sec:consopt} and describe the general regular solution for these problems based on the gradient calculation. 
\par Next, we discuss briefly the bi-level optimization, which is a parameterized lower-level problem binds variables that appear in the objective of an upper-level problem (Section~\ref{sec:bilevel}). Finally, we give a summary of the numerical optimization in constrained problems in Section~\ref{sec:2summary}.




\section{Theory of Optimization}
\label{sec:theory}
\subsection{Existence of Optimizers}
In optimization, a basic question is to determine the existence of a global minimizer for a given function $f$. There are several sufficient conditions on $f$ to guarantee the existence, and the optimizer falls in the feasible set of solutions. For such feasible set, some definitions are following: 
\begin{defn}
    A subset $\Omega \in \mathbb{R}^n$ is called
    \begin{itemize}
        \item \emph{bounded} if there is a constant $R > 0$ such that $\|x\| \leq R$ for all $x \in \Omega$
        \item \emph{closed} if the limit point of any convergent sequence in $\Omega$ always lies in $\Omega$
        \item \emph{compact} if any sequence $\left\{x_{k}\right\}$ in $\Omega$ contains a subsequence that converges to a point in $\Omega$
    \end{itemize}
\end{defn}
The following result gives a characterization of compact sets in $\mathbb{R}$. 
\begin{lemma}[Bolzano-Weierstrass theorem]
    A subset $\Omega$ in $\mathbb{R}^n$ is \emph{compact} if and only if it is bounded and closed.
\end{lemma}





\subsection{Optimality Conditions for Unconstrained Problems}

\subsection{Optimality Conditions for Constrained Problems}

\subsection{Duality}


\section{Unconstrained and Constrained Optimization}
\label{sec:consopt}
\subsection{Unconstrained Optimization}
\subsection{Equality Constrained Optimization}
\subsection{Inequality Constrained Optimization}

\section{Bi-level Optimization}
\label{sec:bilevel}



\section{Summary}
\label{sec:2summary}
Summary what you discussed in this chapter, and mention the story in next
chapter. Readers should roughly understand what your thesis takes about by only reading
words at the beginning and the end (Summary) of each chapter.



